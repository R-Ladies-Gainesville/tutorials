% Options for packages loaded elsewhere
\PassOptionsToPackage{unicode}{hyperref}
\PassOptionsToPackage{hyphens}{url}
%
\documentclass[
]{article}
\usepackage{amsmath,amssymb}
\usepackage{lmodern}
\usepackage{ifxetex,ifluatex}
\ifnum 0\ifxetex 1\fi\ifluatex 1\fi=0 % if pdftex
  \usepackage[T1]{fontenc}
  \usepackage[utf8]{inputenc}
  \usepackage{textcomp} % provide euro and other symbols
\else % if luatex or xetex
  \usepackage{unicode-math}
  \defaultfontfeatures{Scale=MatchLowercase}
  \defaultfontfeatures[\rmfamily]{Ligatures=TeX,Scale=1}
\fi
% Use upquote if available, for straight quotes in verbatim environments
\IfFileExists{upquote.sty}{\usepackage{upquote}}{}
\IfFileExists{microtype.sty}{% use microtype if available
  \usepackage[]{microtype}
  \UseMicrotypeSet[protrusion]{basicmath} % disable protrusion for tt fonts
}{}
\makeatletter
\@ifundefined{KOMAClassName}{% if non-KOMA class
  \IfFileExists{parskip.sty}{%
    \usepackage{parskip}
  }{% else
    \setlength{\parindent}{0pt}
    \setlength{\parskip}{6pt plus 2pt minus 1pt}}
}{% if KOMA class
  \KOMAoptions{parskip=half}}
\makeatother
\usepackage{xcolor}
\IfFileExists{xurl.sty}{\usepackage{xurl}}{} % add URL line breaks if available
\IfFileExists{bookmark.sty}{\usepackage{bookmark}}{\usepackage{hyperref}}
\hypersetup{
  pdftitle={Getting Around GitHub},
  pdfauthor={Michelle Slawinski},
  hidelinks,
  pdfcreator={LaTeX via pandoc}}
\urlstyle{same} % disable monospaced font for URLs
\usepackage[margin=1in]{geometry}
\usepackage{color}
\usepackage{fancyvrb}
\newcommand{\VerbBar}{|}
\newcommand{\VERB}{\Verb[commandchars=\\\{\}]}
\DefineVerbatimEnvironment{Highlighting}{Verbatim}{commandchars=\\\{\}}
% Add ',fontsize=\small' for more characters per line
\usepackage{framed}
\definecolor{shadecolor}{RGB}{248,248,248}
\newenvironment{Shaded}{\begin{snugshade}}{\end{snugshade}}
\newcommand{\AlertTok}[1]{\textcolor[rgb]{0.94,0.16,0.16}{#1}}
\newcommand{\AnnotationTok}[1]{\textcolor[rgb]{0.56,0.35,0.01}{\textbf{\textit{#1}}}}
\newcommand{\AttributeTok}[1]{\textcolor[rgb]{0.77,0.63,0.00}{#1}}
\newcommand{\BaseNTok}[1]{\textcolor[rgb]{0.00,0.00,0.81}{#1}}
\newcommand{\BuiltInTok}[1]{#1}
\newcommand{\CharTok}[1]{\textcolor[rgb]{0.31,0.60,0.02}{#1}}
\newcommand{\CommentTok}[1]{\textcolor[rgb]{0.56,0.35,0.01}{\textit{#1}}}
\newcommand{\CommentVarTok}[1]{\textcolor[rgb]{0.56,0.35,0.01}{\textbf{\textit{#1}}}}
\newcommand{\ConstantTok}[1]{\textcolor[rgb]{0.00,0.00,0.00}{#1}}
\newcommand{\ControlFlowTok}[1]{\textcolor[rgb]{0.13,0.29,0.53}{\textbf{#1}}}
\newcommand{\DataTypeTok}[1]{\textcolor[rgb]{0.13,0.29,0.53}{#1}}
\newcommand{\DecValTok}[1]{\textcolor[rgb]{0.00,0.00,0.81}{#1}}
\newcommand{\DocumentationTok}[1]{\textcolor[rgb]{0.56,0.35,0.01}{\textbf{\textit{#1}}}}
\newcommand{\ErrorTok}[1]{\textcolor[rgb]{0.64,0.00,0.00}{\textbf{#1}}}
\newcommand{\ExtensionTok}[1]{#1}
\newcommand{\FloatTok}[1]{\textcolor[rgb]{0.00,0.00,0.81}{#1}}
\newcommand{\FunctionTok}[1]{\textcolor[rgb]{0.00,0.00,0.00}{#1}}
\newcommand{\ImportTok}[1]{#1}
\newcommand{\InformationTok}[1]{\textcolor[rgb]{0.56,0.35,0.01}{\textbf{\textit{#1}}}}
\newcommand{\KeywordTok}[1]{\textcolor[rgb]{0.13,0.29,0.53}{\textbf{#1}}}
\newcommand{\NormalTok}[1]{#1}
\newcommand{\OperatorTok}[1]{\textcolor[rgb]{0.81,0.36,0.00}{\textbf{#1}}}
\newcommand{\OtherTok}[1]{\textcolor[rgb]{0.56,0.35,0.01}{#1}}
\newcommand{\PreprocessorTok}[1]{\textcolor[rgb]{0.56,0.35,0.01}{\textit{#1}}}
\newcommand{\RegionMarkerTok}[1]{#1}
\newcommand{\SpecialCharTok}[1]{\textcolor[rgb]{0.00,0.00,0.00}{#1}}
\newcommand{\SpecialStringTok}[1]{\textcolor[rgb]{0.31,0.60,0.02}{#1}}
\newcommand{\StringTok}[1]{\textcolor[rgb]{0.31,0.60,0.02}{#1}}
\newcommand{\VariableTok}[1]{\textcolor[rgb]{0.00,0.00,0.00}{#1}}
\newcommand{\VerbatimStringTok}[1]{\textcolor[rgb]{0.31,0.60,0.02}{#1}}
\newcommand{\WarningTok}[1]{\textcolor[rgb]{0.56,0.35,0.01}{\textbf{\textit{#1}}}}
\usepackage{longtable,booktabs,array}
\usepackage{calc} % for calculating minipage widths
% Correct order of tables after \paragraph or \subparagraph
\usepackage{etoolbox}
\makeatletter
\patchcmd\longtable{\par}{\if@noskipsec\mbox{}\fi\par}{}{}
\makeatother
% Allow footnotes in longtable head/foot
\IfFileExists{footnotehyper.sty}{\usepackage{footnotehyper}}{\usepackage{footnote}}
\makesavenoteenv{longtable}
\usepackage{graphicx}
\makeatletter
\def\maxwidth{\ifdim\Gin@nat@width>\linewidth\linewidth\else\Gin@nat@width\fi}
\def\maxheight{\ifdim\Gin@nat@height>\textheight\textheight\else\Gin@nat@height\fi}
\makeatother
% Scale images if necessary, so that they will not overflow the page
% margins by default, and it is still possible to overwrite the defaults
% using explicit options in \includegraphics[width, height, ...]{}
\setkeys{Gin}{width=\maxwidth,height=\maxheight,keepaspectratio}
% Set default figure placement to htbp
\makeatletter
\def\fps@figure{htbp}
\makeatother
\setlength{\emergencystretch}{3em} % prevent overfull lines
\providecommand{\tightlist}{%
  \setlength{\itemsep}{0pt}\setlength{\parskip}{0pt}}
\setcounter{secnumdepth}{-\maxdimen} % remove section numbering
\ifluatex
  \usepackage{selnolig}  % disable illegal ligatures
\fi

\title{Getting Around GitHub}
\author{Michelle Slawinski}
\date{3/6/2022}

\begin{document}
\maketitle

{
\setcounter{tocdepth}{2}
\tableofcontents
}
\hypertarget{welcome}{%
\section{Welcome}\label{welcome}}

Git.. GitHub.. What are they? Why do they matter? In this tutorial we
will introduce you to them and provide you with the basics! This
presentation is for individuals with limited knowledge on git and
graphical user interfaces, although those with a robust background are
more than welcome to drop some knowledge on us! From one novice git user
to another, we will walk through it together.

\hypertarget{learning-objectives}{%
\section{Learning Objectives}\label{learning-objectives}}

\begin{enumerate}
\def\labelenumi{\arabic{enumi}.}
\tightlist
\item
  Describe git and graphical user interfaces
\item
  Set up and connect git, GitHub and RStudio
\item
  Practice navigating how to fork, commit, and request a pull on Github
\end{enumerate}

\hypertarget{git}{%
\section{Git}\label{git}}

\hypertarget{what-is-git}{%
\subsection{What is Git?}\label{what-is-git}}

Have you ever tried working on a project or a specific piece of code
with a colleague but could not figure out what changes they made? Or
have you ever overwritten your own code and had to rewrite the chunk
from the beginning?

Frustrating, right? Well, git is a version control that allows you to do
better. Be a better coder, communicator and person because the time you
will save can then be spent elsewhere!

\hypertarget{set-up}{%
\subsection{Set Up}\label{set-up}}

Let's get started shall we!

First, check to see if you already have Git installed.

On Windows, look for an application called ``Git Bash'' and on Macs,
search for an application called ``Terminal.'' Type ``git version'' into
the terminal. If you receive a version number, Git is installed.
However, if the terminal is saying unknown command, then Git it not
installed.

No Git? No problem. Navigate to \href{https://git-scm.com/}{git} and
click \emph{Downloads.} Select the software appropriate for your system
and follow the remaining installation steps.

\textbf{Windows}

\begin{enumerate}
\def\labelenumi{\arabic{enumi}.}
\tightlist
\item
  Select \textbf{``Click here to download''} This should be the version
  you need however, you can look below to see additional options.
\item
  Allow installer to download then open.
\item
  ``Do you want to allow this app to make changes to this device?'' Yes,
  yes you do! You will need administrative access to your computer for
  this.
\item
  Use the defaults throughout the installation process.
\end{enumerate}

\textbf{Mac}

If you have Git installed, then we will most likely just need to update
it. To update git follow this
\href{https://modulesunraveled.com/installing-git/updating-git-if-you-have-only-version-comes-xcode-or-command-line-developer-tools}{video}!

Otherwise, to install git:

\begin{enumerate}
\def\labelenumi{\arabic{enumi}.}
\tightlist
\item
  Select \textbf{``Click here to download.''} This should be the version
  you need however, you can look below to see additional options.
\item
  Click on the Binary installer link and select download.
\item
  Once it is downloaded, open the file, locate the package, right click
  and select ``Open.''
\item
  Follow the prompts.
\item
  For help with the remaining installation process, follow this
  \href{https://modulesunraveled.com/installing-git/installing-git-if-you-do-not-have-xcode-or-command-line-developer-tools-installed}{video}.
\end{enumerate}

\hypertarget{git-configuration-basics}{%
\subsection{Git Configuration Basics}\label{git-configuration-basics}}

When you configure git you will have options on what level you would
like to make configurations.

\begin{verbatim}
 System Level: git config --system

 User Level: git config --global

 Project Level: git config
\end{verbatim}

Let's set up some basic git configurations for our system.

\begin{enumerate}
\def\labelenumi{\arabic{enumi}.}
\tightlist
\item
  Username: When you set your user name using global, git will use this
  information for anything you do.

  \begin{itemize}
  \tightlist
  \item
    git config -\/-global user.name ``Michelle Slawinski''
  \end{itemize}
\item
  Email: Do the same thing with your email address. This should be the
  email you used to set up your GitHub account. If you have not signed
  up for a GitHub account, do that now.

  \begin{itemize}
  \tightlist
  \item
    git config -\/-global user.email ``my\_email.com''
  \end{itemize}
\item
  Viewing git configurations: Type this command and scroll to the bottom
  to see the configurations you set up.

  \begin{itemize}
  \tightlist
  \item
    git config -\/-list
  \end{itemize}
\end{enumerate}

For other configuration options see
\href{https://git-scm.com/docs/git-config}{here}.

\hypertarget{help-options}{%
\subsection{Help Options}\label{help-options}}

Are you stuck? Ask Git! In the terminal, type ``git help'' and it will
return some common Git commands. If you want to learn more about a
specific command such as branching, you can type ``git help branch'' and
it will pull up a manual on branching.

\begin{figure}
\centering
\includegraphics{git help branch.PNG}
\caption{image}
\end{figure}

\hypertarget{git-resources}{%
\subsection{Git Resources}\label{git-resources}}

\begin{itemize}
\tightlist
\item
  \href{https://git-scm.com/book/en/v2}{Pro Git Book}
\item
  \href{https://git-scm.com/docs}{Git Reference Guides}
\item
  \href{https://github.com/git-guides/install-git}{GitHub Git Guides}
\item
  \href{https://github.com/dictcp/awesome-git}{`Awesome Git' Repository}
\item
  \href{https://www.linkedin.com/learning-login/share?account=42166124\&forceAccount=false\&redirect=https\%3A\%2F\%2Fwww.linkedin.com\%2Flearning\%2Fgit-essential-training-the-basics\%3Ftrk\%3Dshare_ent_url\%26shareId\%3DSxKxywffQ8iDu\%252FzDfk84uw\%253D\%253D}{LinkedIn
  Learning Git Essential Training: The Basics}
\end{itemize}

\hypertarget{graphical-user-interface}{%
\section{Graphical User Interface}\label{graphical-user-interface}}

\hypertarget{what-is-a-graphical-user-interface}{%
\subsection{What is a Graphical User
Interface?}\label{what-is-a-graphical-user-interface}}

Git but better. Graphical User Interfaces (GUIs) are where you can use
Git but in a more visually pleasing manner. Instead of using command
lines in Git, you can use a click-based GUI. Think of it like using
R/SAS versus SPSS or Jasp.

\hypertarget{popular-guis}{%
\subsection{Popular GUIs}\label{popular-guis}}

\begin{longtable}[]{@{}ll@{}}
\toprule
Microsoft & Mac \\
\midrule
\endhead
GitHub Desktop & GitHub Desktop \\
GitKraken & GitKraken \\
Sourcetree & Sourcetree \\
TortoiseGit & Fork \\
SmartGit & \\
\bottomrule
\end{longtable}

\hypertarget{gui-resources}{%
\subsection{GUI Resources}\label{gui-resources}}

\begin{itemize}
\tightlist
\item
  \href{https://git-scm.com/downloads/guis}{GUI Clients}
\item
  \href{https://docs.github.com/en}{GitHub Help}
\item
  \href{https://support.gitkraken.com/}{GitKraken Support}
\item
  \href{https://confluence.atlassian.com/get-started-with-sourcetree}{Sourcetree
  Support}
\end{itemize}

\hypertarget{github}{%
\section{GitHub}\label{github}}

\hypertarget{what-is-github}{%
\subsection{What is GitHub?}\label{what-is-github}}

GitHub makes it easier on us as human beings! GitHub is a cloud based
management tool for your code. Think of it like Google Docs' tracked
changes feature. In any document, you can use tracked changes to
\emph{visualize} the changes you or someone else are making. In GitHub,
we are using a similar way to visualize version control. This is
extremely important when you start working in teams or on a large
project. You will want to have changes to your code saved so that you
can refer back if you need to.

\hypertarget{setup}{%
\subsection{Setup}\label{setup}}

Navigate \href{https://desktop.github.com}{here} to download GitHub
Desktop and follow prompts.

Note: You will want to create an account and hold on to the email you
use for this account.

\hypertarget{lets-practice}{%
\subsection{Let's Practice}\label{lets-practice}}

Now that you have an understanding of Git and GUIs such as GitHub, let's
practice!

We will be practicing how to fork a repository from GitHub, make a
change locally, then commit those changes and push them back to GitHub.
Ready? Lets get our feet wet.

\begin{enumerate}
\def\labelenumi{\arabic{enumi}.}
\tightlist
\item
  Navigate to \href{https://github.com/R-Ladies-Gainesville}{RLadies of
  Gainesville's GitHub}
\item
  Locate and click on the ``practice'' repository
\item
  In the upper right hand corner, click ``Fork'' Once you have it
  forked, you will see that it is in your remote GitHub. Now that you
  have it remotely, we will want to get access to it locally so that we
  can add to the repository.
\item
  Click into the `practice' repository if you are not already in it and
  click on the 'Code" drop down.
\item
  Select Open with GitHub Desktop. Once you click this, GitHub Desktop
  should open with a screen asking you where you would like to place the
  repository on your desktop. Go ahead and find where you would like to
  put the repository.
\end{enumerate}

\hypertarget{rstudio}{%
\section{RStudio}\label{rstudio}}

\hypertarget{setup-1}{%
\subsection{Setup}\label{setup-1}}

\begin{Shaded}
\begin{Highlighting}[]
\FunctionTok{library}\NormalTok{(usethis)}
\end{Highlighting}
\end{Shaded}

You should have the latest version of R and Rstudio downloaded. To check
your current version of R:

\begin{Shaded}
\begin{Highlighting}[]
\NormalTok{R.version.string}
\CommentTok{\#\textgreater{} [1] "R version 4.0.5 (2021{-}03{-}31)"}
\end{Highlighting}
\end{Shaded}

You will also want to have git installed at this point. To introduce
yourself to git using RStudio:

\begin{Shaded}
\begin{Highlighting}[]
\FunctionTok{use\_git\_config}\NormalTok{(}\AttributeTok{user.name =} \StringTok{"Michelle Slawinski"}\NormalTok{, }\AttributeTok{user.email =} \StringTok{"slawinsm@gmail.com"}\NormalTok{) }
  \CommentTok{\#use.email should be the same as your GitHub account}
\end{Highlighting}
\end{Shaded}

To configure GitHub with RStudio, you will need to set up a token in
RStudio. This is basically a secure connection between GitHub and
RStudio:

\begin{Shaded}
\begin{Highlighting}[]
\NormalTok{usethis}\SpecialCharTok{::}\FunctionTok{create\_github\_token}\NormalTok{()}
\CommentTok{\#\textgreater{} * Call \textasciigrave{}gitcreds::gitcreds\_set()\textasciigrave{} to register this token in the local Git credential store}
\CommentTok{\#\textgreater{}   It is also a great idea to store this token in any password{-}management software that you use}
\CommentTok{\#\textgreater{} * Open URL \textquotesingle{}https://github.com/settings/tokens/new?scopes=repo,user,gist,workflow\&description=DESCRIBE THE TOKEN\textbackslash{}\textquotesingle{}S USE CASE\textquotesingle{}}
\NormalTok{gitcreds}\SpecialCharTok{::}\NormalTok{gitcreds\_set}
\CommentTok{\#\textgreater{} function (url = "https://github.com") }
\CommentTok{\#\textgreater{} \{}
\CommentTok{\#\textgreater{}     if (!is\_interactive()) \{}
\CommentTok{\#\textgreater{}         throw(new\_error("gitcreds\_not\_interactive\_error", message = "\textasciigrave{}gitcreds\_set()\textasciigrave{} only works in interactive sessions"))}
\CommentTok{\#\textgreater{}     \}}
\CommentTok{\#\textgreater{}     stopifnot(is\_string(url), has\_no\_newline(url))}
\CommentTok{\#\textgreater{}     check\_for\_git()}
\CommentTok{\#\textgreater{}     current \textless{}{-} tryCatch(gitcreds\_get(url, use\_cache = FALSE, }
\CommentTok{\#\textgreater{}         set\_cache = FALSE), gitcreds\_no\_credentials = function(e) NULL)}
\CommentTok{\#\textgreater{}     if (!is.null(current)) \{}
\CommentTok{\#\textgreater{}         gitcreds\_set\_replace(url, current)}
\CommentTok{\#\textgreater{}     \}}
\CommentTok{\#\textgreater{}     else \{}
\CommentTok{\#\textgreater{}         gitcreds\_set\_new(url)}
\CommentTok{\#\textgreater{}     \}}
\CommentTok{\#\textgreater{}     msg("{-}\textgreater{} Removing credetials from cache...")}
\CommentTok{\#\textgreater{}     gitcreds\_delete\_cache(gitcreds\_cache\_envvar(url))}
\CommentTok{\#\textgreater{}     msg("{-}\textgreater{} Done.")}
\CommentTok{\#\textgreater{}     invisible()}
\CommentTok{\#\textgreater{} \}}
\CommentTok{\#\textgreater{} \textless{}bytecode: 0x0000000013d2aa10\textgreater{}}
\CommentTok{\#\textgreater{} \textless{}environment: 0x00000000131fa7e0\textgreater{}}
\end{Highlighting}
\end{Shaded}

\hypertarget{rstudio-resources}{%
\subsection{RStudio Resources}\label{rstudio-resources}}

\begin{itemize}
\tightlist
\item
  \href{https://r-bio.github.io/intro-git-rstudio/}{An Introduction to
  Git and how to use it with RStudio}
\item
  \href{https://resources.github.com/whitepapers/github-and-rstudio/}{GitHub
  and RStudio}
\item
  \href{https://happygitwithr.com/connect-intro.html}{Happy Git and
  GitHub for the useR}
\end{itemize}

\hypertarget{last-minute-things}{%
\section{Last Minute Things}\label{last-minute-things}}

\hypertarget{dictionary}{%
\subsection{Dictionary}\label{dictionary}}

\begin{longtable}[]{@{}
  >{\raggedright\arraybackslash}p{(\columnwidth - 2\tabcolsep) * \real{0.47}}
  >{\raggedright\arraybackslash}p{(\columnwidth - 2\tabcolsep) * \real{0.53}}@{}}
\toprule
Term & Definition \\
\midrule
\endhead
Shell & A terminal that uses written commands to interact with your
operating system \\
Git Directory & Umbrella terms for all of your files and folders \\
Modify & Changed the file but have not committed it to your local
system \\
Staged/Indexed & Marked a modified file to go into your next commit
snapshot \\
Commit & Save the state of your project in your local system \\
Branch & When you duplicate part of a repository \\
Merge & After you finish updating the code you branched, you can merge
it back into the project's main source code \\
Push & Once you commit your project or file to your local system, you
have the option to push it to remote GitHub \\
Pull & If your repository has new changes in GitHub, you can download
and integrate those remote changes by using `pull' \\
Fetch & To see if there are remote changes, you can use fetch to see
what was updated. Fetch will not merge your updates though \\
\bottomrule
\end{longtable}

\hypertarget{tips}{%
\subsection{Tips}\label{tips}}

\begin{enumerate}
\def\labelenumi{\arabic{enumi}.}
\tightlist
\item
  Update some basic profile features GitHub offers like a picture,
  occupation, or location! Click your picture in the top right corner
  after logging in and select `Settings.'
\item
  You can also personalize your profile's ReadMe file with an
  introduction for people who will visit your page. This is again
  optional but helps you get noticed. If you want to see some super cool
  examples, beef up your profile with certain tools, or do some late
  night reading on how to make your profile stand out, look at this
  \href{https://github.com/abhisheknaiidu/awesome-github-profile-readme}{repository}!
\item
  Make sure you have a ReadMe file for every repository. This serves as
  an executive summary for your project and saves viewers time on
  figuring out what the purpose of your project is.
\item
  When you make a commit, make sure you are descriptive, especially if
  you are working on a team. You can use git to search your commit
  messages so you'll want to be descriptive! Google `git commit best
  practices' and you can look through all the suggestions on how to do
  this.
\item
  Branch this markdown file and add your own tips!
\end{enumerate}

\& that's it!

\end{document}
